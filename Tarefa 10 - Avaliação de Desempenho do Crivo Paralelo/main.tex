\documentclass{article}
\usepackage[utf8]{inputenc}
\usepackage[brazil]{babel}
\usepackage{setspace}
\usepackage{mathtools}
\usepackage{pgfplots}
\usepackage{listings}
\usepackage{xcolor}
\usepackage{natbib}
\usepackage{graphicx}
\usepackage{array}

\DeclarePairedDelimiter\ceil{\lceil}{\rceil}
\DeclarePairedDelimiter\floor{\lfloor}{\rfloor}

\definecolor{codegreen}{rgb}{0,0.6,0}
\definecolor{codegray}{rgb}{0.5,0.5,0.5}
\definecolor{codepurple}{rgb}{0.58,0,0.82}
\definecolor{backcolour}{rgb}{0.95,0.95,0.92}

\lstdefinestyle{codigo}{
    numberstyle=\tiny,
    basicstyle=\ttfamily\footnotesize,
    breakatwhitespace=false,         
    breaklines=true,                 
    captionpos=b,                    
    keepspaces=true,                 
    numbers=left,                    
    numbersep=5pt,                  
    showspaces=false,                
    showstringspaces=false,
    showtabs=false,                  
    tabsize=2
}

\lstset{style=codigo}

\setstretch{1.5}
\pgfplotsset{width=10cm,compat=1.9}

\pagenumbering{gobble}
\clearpage
\thispagestyle{empty}

\title{Avaliando ao Desempenho do Crivo Paralelo - Ferramenta \emph{Perf}}
\author{Lucas Santiago}
\date{Março de 2022}

\renewcommand{\thefootnote}{*}

\begin{document}
\maketitle

    \vspace{2cm}

    \section*{Código do Crivo de Erastóteles}
    \lstinputlisting[language=c]{src/crivoDeEratostenes.c}

    
    \newpage
    \section*{Valores obtidos pela ferramenta \emph{perf}}
    \begin{table}[ht]
        \centering
        \begin{tabular}{|m{4cm}|c|c|}
            \hline
            & Crivo Sequêncial & Crivo Paralelo \\
            \hline
            CPU utilizada (\%) & 99,8\% & 126\% \\
            \hline
            Ciclos ociosos na ULA & \footnote[1]{} & \footnote[1]{} \\
            \hline
            Ciclos ociosos na busca de instrução & \footnote[1]{} & \footnote[1]{} \\
            \hline
            Trocas de contexto & 0,085 K/sec & 0,085 K/sec \\
            \hline
            
            \emph{Page Faults} & 0,006 M/sec & 0,005 M/sec \\
            \hline
            
            Instruções por ciclo & 0,38 inst por cicle & 0,34 inst por cicle \\
            \hline
            Taxa de falta na cache L3 & 1,419 M/sec & 2,117 M/sec \\
            \hline
            Tempo total de execução & 4,072374329 segundos & 3,846906854 segundos \\
            \hline
        \end{tabular}
        \caption{Tabela de desempenho do Crivo de Erastóteles}
    \end{table}
    \footnotetext[1]{Elemento não presente na ferramenta \emph{Perf}}
    
    \section*{Possíveis ideias de otimização do código}
    Olhando para esses valores é possível notar que a execução do código sequên- cial consegue
    fazer um grande uso de 99,8\% da CPU utilizada, enquanto a versão paralela está tendo um ganho de
    aproximadamente 26\% a mais na segunda CPU utilizada. Com esse número é fácil de notar uma falta de balanceamento
    de carga de trabalho entre as \emph{threads}. A primeira consegue fazer um uso de quase 100\% da CPU,
    enquanto a segunda tem apenas $1/4$ de sua capacidade utilizada. Uma primeira otimização seria dividir melhor
    o trabalho entre as \emph{threads}, usando uma \emph{scheduler} melhor no grande for do crivo.
    
    Apenas isso não será suficiente, outro ponto notado foi a gigantesca taxa de erro
    da cache L3. Um aumento em aproximadamente 49\% da versão sequêncial para a paralela.
    Uma melhor distribuição das variáveis dentro da cache seria um ganho significativo.
    Para isso um melhor uso de variáveis do tipo \emph{shared} e \emph{private} teria 
    um possível ganho.  


\end{document}